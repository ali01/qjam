\documentclass[%
        %draft,
        %submission,
        %compressed,
        final,
        %
        %technote,
        %internal,
        %submitted,
        %inpress,
        %reprint,
        %
        %titlepage,
        notitlepage,
        %anonymous,
        narroweqnarray,
        inline,
        %twoside,
        ]{ieee}
%
% some standard modes are:
%
% \documentclass[draft,narroweqnarray,inline]{ieee}
% \documentclass[submission,anonymous,narroweqnarray,inline]{ieee}
% \documentclass[final,narroweqnarray,inline]{ieee}

% Use the `endfloat' package to move figures and tables to the end
% of the paper. Useful for `submission' mode.
%\usepackage {endfloat}

% Use the `times' package to use Helvetica and Times-Roman fonts
% instead of the standard Computer Modern fonts. Useful for the
% IEEE Computer Society transactions.
% (Note: If you have the commercial package `mathtime,' it is much
% better, but the `times' package works too).
%\usepackage {times}

% In order to use the figure-defining commands in ieeefig.sty...
\usepackage{ieeefig,url}

\begin{document}

%----------------------------------------------------------------------
% Title Information, Abstract and Keywords
%----------------------------------------------------------------------
\title[Scaling Machine Learning Algorithms]{%
       Scaling Machine Learning Algorithms}

% format author this way for journal articles.
\author[SHORT NAMES]{
  Juan Batiz-Benet \\
  \and{\quad\quad}
  Quinn Slack \\
  \and{\quad\quad}
  Matt Sparks \\
  \and{\quad\quad}
  Ali Yahya
}

% specifiy the journal name
%\journal{IEEE Transactions on Something, 1997}

% Or, when the paper is a preprint, try this...
%\journal{IEEE Transactions on Something, 1997, TN\#9999.}

% Or, specify the conference place and date.
%\confplacedate{Ottawa, Canada, May 19--21, 1997}

% make the title
\maketitle

% do the abstract
\begin{abstract}
Our premise is ...
\end{abstract}

% do the keywords
\begin{keywords}
keyword 1, keyword 2 ...
\end{keywords}

\section{Introduction}

\PARstart Use the PARstart macro only for the first paragraph in
the entire paper...

% try out a theorem...
\newtheorem{theorem}{Theorem}

\begin{theorem}[Theorem name]
  Consider the system ...
\end{theorem}

\begin{proof}
  The proof is trivial.
\end{proof}

\section{Another Section}

foo bar

% do the biliography:
%\bibliographystyle{IEEEbib}
%\bibliography{my-bibliography-file}

%----------------------------------------------------------------------
% FIGURES
%----------------------------------------------------------------------
% There are many ways to include figures in the text. We will assume
% that the figure is some sort of EPS file.
%
% The outdated packages epsfig and psfig allow you to insert figures
% like: \psfig{filename.eps} These should really be done now using the
% \includegraphics{filename.eps} command.
%
% i.e.,
%
% \includegraphics{file.eps}
%
% whenever you want to include the EPS file 'file.eps'. There are many
% options for the includegraphics command, and are outlined in the
% on-line documentation for the "graphics bundle". Using the options,
% you can specify the height, total height (height+depth), width, scale,
% angle, origin, bounding box "bb",view port, and can trim from around
% the sides of the figure. You can also force LaTeX to clip the EPS file
% to the bounding box in the file. I find that I often use the scale,
% trim and clip commands.
%
% \includegraphics[scale=0.6,trim=0 0 0 0,clip=]{file.eps}
%
% which magnifies the graphics by 0.6 (If I create a graphics for an
% overhead projector transparency, I find that a magnification of 0.6
% makes it look much better in a paper), trims 0 points off
% of the left, bottom, right and top, and clips the graphics. If the
% trim numbers are negative, space is added around the figure. This can
% be useful to help center the graphics, if the EPS file bounding box is
% not quite right.
%
% To center the graphics,
%
% \begin{center}
% \includegraphics...
% \end{center}
%
% I have not yet written good documentation for this, but another
% package which helps in figure management is the package ieeefig.sty,
% available at: http://www-isl.stanford.edu/people/glp/ieee.shtml
% Specify:
%
%\usepackage{ieeefig}
%
% in the preamble, and whenever you want a figure,
%
%\figdef{filename}
%
% where, filename.tex is a LaTeX file which defines what the figure is.
% It may be as simple as
%
% \inserteps{filename.eps}
%
% or
% \inserteps[includegraphics options]{filename.eps}
%
% or may be a very complicated LaTeX file.

\end{document}
