\documentclass[11pt]{article}

\usepackage[top=1in, bottom=1in, left=0.8in, right=1in]{geometry}
\usepackage{wrapfig}
\usepackage{graphicx}
\usepackage{amsmath}
\usepackage{listings}


\title{R vs Python \\ linear algebra benchmarks }
\author{
\begin{tabular}{rl}
  Juan Batiz-Benet & jbenet@cs.stanford.edu \\
  Quinn Slack      & sqs@cs.stanford.edu \\
  Matt Sparks      & msparks@cs.stanford.edu \\
  Ali Yahya        & alive@cs.stanford.edu \\
\end{tabular}}
\date{\today}

\begin{document}

\maketitle

 In order to determine the viability of the R Programming Language for efficient computation of machine learning algorithms, we compared benchmarks against Python (using the numpy package). In order to avoid implementation or algorithm-specific bias, we decided to benchmark common linear algebra functions (e.g. matrix multiplication) ubiquitous in learning algorithms.

The following tables show the running times of python and R on various operations using different sizes, as well as the time ratio of python / R. Interestingly, python outperforms R in every case, most of the time by an order of magnitude. In the worst case, Python takes 0.8 the time of R.

\begin{center}
    \textbf{Matrix Addition} \\
\begin{tabular}{cccc}
size  & python  &  R       & python / R \\
  \hline
50  & 0.0060 & 0.0159 & 0.3774 \\
75  & 0.0110 & 0.0302 & 0.3642 \\
100 & 0.0170 & 0.0519 & 0.3276 \\
150 & 0.0350 & 0.1161 & 0.3015 \\
250 & 0.0950 & 0.3396 & 0.2797 \\
\end{tabular}

    \textbf{Matrix Multiplication} \\
\begin{tabular}{cccc}
size  & python  &  R       & python / R \\
  \hline
50  & 0.1600  & 0.2208  & 0.7246 \\
75  & 0.5800  & 0.7339  & 0.7903 \\
100 & 1.3030  & 1.6323  & 0.7983 \\
150 & 4.2350  & 5.2311  & 0.8096 \\
250 & 18.9190 & 22.9759 & 0.8234 \\
\end{tabular}

    \textbf{Element-Wise Matrix Multiplication} \\
\begin{tabular}{cccc}
size  & python  &  R       & python / R \\
  \hline
150   & 0.0350  &  0.1576  &  0.2221 \\
225   & 0.0760  &  0.3741  &  0.2032 \\
300   & 0.1510  &  0.6859  &  0.2201 \\
450   & 0.9310  &  2.0938  &  0.4446 \\
750   & 3.3010  &  5.4117  &  0.6100 \\
\end{tabular}

    \textbf{Matrix Transpose} \\
\begin{tabular}{cccc}
size  & python  &  R       & python / R \\
  \hline
50  & 0.0010  & 0.0325 & 0.0308 \\
75  & 0.0010  & 0.0610 & 0.0164 \\
100 & 0.0010  & 0.1030 & 0.0097 \\
150 & 0.0010  & 0.2196 & 0.0046 \\
250 & 0.0010  & 0.6119 & 0.0016 \\
\end{tabular}

\textbf{Vector inner product} \\
\begin{tabular}{cccc}
size  & python  &  R       & python / R \\
  \hline
2500  & 0.0040 & 0.0523 & 0.0765 \\
3750  & 0.0060 & 0.0772 & 0.0777 \\
5000  & 0.0070 & 0.1030 & 0.0680 \\
7500  & 0.0100 & 0.1519 & 0.0658 \\
12500 & 0.0160 & 0.2514 & 0.0636 \\
\end{tabular}

\end{center}



\end{document}

